\documentclass[12pt,a4]{article}

\usepackage[utf8]{inputenc}
\usepackage[english]{babel}
\usepackage{amsmath}
\usepackage{amsfonts}
\usepackage{listings}
\usepackage{fullpage}
\usepackage{booktabs}

\newcommand*{\field}[1]{\mathbb{#1}}
\newcommand*{\R}{\field{R}} % real R
\newcommand*{\codemat}[1]{\texttt{#1}}
\newcommand*{\matpower}{\textsc{matpower}}

\title{Computation of Maximum loadability limit in \matpower}
\author{Camille Hamon}

\begin{document}
\maketitle

\section{Notations}
\label{sec:notations}

If not indicated otherwise, the notations are the same as in \matpower's manual.

\section{The maximum loadability limit problem}
\label{sec:maxim-load-limit}

Finding the maximum loadability limit from base-case loading $P_d^0 \in \R^{n_b}$ in a load increase direction $d \in \R^{n_b}$ can be done by solving the following optimisation problem

\begin{subequations}\label{eq:maxll}
\begin{align}
  \max_{z} \quad & \alpha \label{eq:maxll-obj} \\
  \text{subject to} \quad & P_d = P_d^0 + \alpha d \label{eq:loadinc}\\
   & Q_d = \text{diag}(\tan \phi^0) P_d \label{eq:loadinc-q}\\
   & g(z) = 0 \\
   & z_\text{min} \leq z \leq z_\text{max} \label{eq:zlim}
\end{align}  
\end{subequations}
where
\begin{align}
  \label{eq:z}
  z = [x^T \; \alpha]^T =  \begin{bmatrix}
    \Theta \\
    V_m \\
    P_g \\
    Q_g \\
    \alpha
  \end{bmatrix},
\end{align}
where $x$ is defined in Equation (6.5) in \matpower's manual and the function $g$ is defined is Equations (4.2) and (4.3) in \matpower's manual.
Equation \eqref{eq:loadinc-q} ensures that the power factor of the loads is kept constant to $\cos(\phi_i^0)$, the value defined in the base case.
Note that $\tan \phi^0 \in \R^{n_b}$.
The limits \eqref{eq:zlim} are as follows:
\begin{align}
  & \theta_i = \theta_i^\text{ref}, && i \in \mathcal{I}_\text{ref} \label{eq:thetaslack} \\
  & v_m^{i} = v_m^{i,\text{ref}}, && i \in \mathcal{I}_\text{ref} \label{eq:vmslack} \\
  0 \leq & v_m^{i} \leq v_m^{i,\text{ref}}, && i \in \{1,\ldots,n_g\} \backslash \mathcal{I}_\text{ref} \label{eq:vmpv} \\
  q_q^{i,\text{min}} \leq & q_g^i \leq q_g^{i,\text{max}}, && i \in \{1,\ldots,n_g\} \backslash \mathcal{I}_\text{ref} \label{eq:qgpv}\\
  & p_g^i = p_g^{i,0}, && i \in \{1,\ldots,n_g\} \backslash \mathcal{I}_\text{ref} \\
  & \alpha \in [0,+\infty].
\end{align}
where $p_g^{i,0}$ is the active power generation at bus $i$ in the base case.
Equations \eqref{eq:thetaslack} and \eqref{eq:vmslack} lock the voltage angle and magnitude of the reference buses to their set points.
Equations \eqref{eq:vmpv} and \eqref{eq:qgpv} limit the terminal voltages and reactive power generation at the PV buses. 
Note that in theory, complementarity constraints
\begin{align}
  \label{eq:comp-constr}
  (v_m^i-v_m^{i,\text{ref}})\cdot(q_g^i-q_g^{i,\text{max}}), \quad i \in \{1,\ldots,n_g\} \backslash \mathcal{I}_\text{ref}
\end{align}
should be included. 
However, in practice, solvers will set $v_m^i = v_m^{i,\text{ref}}$ if $q_g^i$ is smaller than $q_g^{i,\text{max}}$ and let $v_m^i$ be free if $q_g^i$ has reached its limit, since doing so maximizes loadability.

\section{Implementation in \matpower}
\label{sec:impl-matp}

\subsection{Inputs}
\label{sec:inputs}

We assume that the following inputs are available:
\begin{enumerate}
\item A \matpower base case \codemat{mpc}.
\item A load increase direction $d \in R^{n_b}$, with the assumption that only nonzero loads in \codemat{mpc} can be increased.
\end{enumerate}

\subsection{Defining dispatchable loads}
\label{sec:defin-disp-loads}

A new \matpower case \codemat{mpc\_vl} is built where all nonzero loads in \codemat{mpc} are transformed to dispatchable loads:
\begin{lstlisting}[language=Matlab]
mpc_vl = load2disp(mpc)  
\end{lstlisting}
Dispatchable loads are described in Section 6.4.2 of \matpower's package.
Note that, conceptually, this means that the active parts of the nonzero loads in $P_d$ enter $P_g$ in \eqref{eq:z} and their reactive parts enter $Q_g$.

\subsection{Adjusting the \matpower case}
\label{sec:adjust-matp-case}

\subsubsection{Power constraints of dispatchable loads}
\label{sec:limits-disp-loads}

In \matpower, dispatchable loads are assumed to maintain a constant power factor. 
This power factor is determined by the values in the columns \codemat{PMIN} and \codemat{QMIN} corresponding to the dispatchable loads in the field \codemat{mpc\_vl.gen}.
When using the function \codemat{load2disp}, these fields are set to the opposite of the active and reactive power parts of the nonzero loads in the base case \codemat{mpc}.
When solving the optimisation problem \eqref{eq:maxll} with \matpower, the columns \codemat{PMIN} and \codemat{QMIN} also constrain the maximum values of dispatchable loads.
Since we are looking for maximum loadability limits, upper values on loads must not be binding.
Therefore, the values \codemat{PMIN} of rows in \codemat{mpc\_vl.gen} corresponding to dispatchable loads is set to a very large negative values and the values \codemat{QMIN} of these dispatchable loads are changed accordingly to keep the constant power factor defined in the base case.

\subsubsection{Power constraints on generators}
\label{sec:constr-gen}


The maximum loadability limit problem differs from a standard AC OPF as defined in Equations (6.1) to (6.4) in \matpower's manual in the following way
\begin{enumerate}
\item Active power production at PV buses is locked to the base case values. Therefore, the columns \codemat{PMIN} and \codemat{PMAX} of these generators are set to their active power production in the base case.
\item Generators at reference buses pick up the loads during the load increase process. The assumption here is that the reference buses correspond to strong generators that do not reach their power limits. Therefore, active and reactive power limits of the generators at reference buses are set to large values.
\end{enumerate}

\subsubsection{Voltage constraints}
\label{sec:volt-constr-at}
Similarly to power constraints discussed above, voltage constraints in the maximum loadability problem differ from a standard AC OPF as follows
\begin{enumerate}
\item Voltage magnitudes at the reference buses are locked at their set point values, see Equation \eqref{eq:vmslack}.
\item Voltage magnitudes at all PV and PQ buses are not constrained from below. The reason for this is that the maximum loadability limit point typically corresponds to an operating point that is outside the normal allowed band on bus voltage magnitude. Therefore, to find this point, lower limits on bus voltage magnitudes of PV and PQ buses must be relaxed.
\item For the same reason, upper bounds on voltage magnitudes of PQ buses are also relaxed.
\item Upper bounds on voltage magnitudes of PV buses are set to the voltage set point of the corresponding generators.
\end{enumerate}

\subsubsection{Branch flows}
\label{sec:branch-flows}

For the same reason as for bus voltages, the branch flow constraints must be relaxed so that they do not become binding during the optimisation process.

\subsubsection{Decision variable $\alpha$ and additional constraints}
\label{sec:defin-vari-alpha}

Compared to the standard AC OPF formulation in \matpower, the formulation in \eqref{eq:maxll} contains an additional decision variable $\alpha$, see \eqref{eq:z}, and the additional constraints \eqref{eq:loadinc}.
Note that constraints \eqref{eq:loadinc-q} does not exist in the standard AC OPF formulation either, but \matpower assumes a constant power factor model for the dispatchable loads, see also Section \ref{sec:limits-disp-loads}.
Therefore, there is no need to define the additional constraint \eqref{eq:loadinc-q}.
The decision variable $\alpha$ and the constraint \eqref{eq:loadinc} are defined using a callback function for the \codemat{formulation} stage.

Following Section 7.2 in \matpower's manual, we create a callback function called \codemat{userfcn\_direction\_mll\_formulation(om,args)} in which we define the new variable and constraints.
To define the new constraint, recall that when transforming the loads to dispatchable loads, the nonzero loads of the base case enter the set of generators, see Section \ref{sec:defin-disp-loads}.
\matpower has a feature to only define the additional constraints on the relevant components of the decision variable $z$ in \eqref{eq:z}.
For Equation \eqref{eq:loadinc}, the relevant component of $z$ are $P_g$ (since the dispatchable loads enter $P_g$) and $\alpha$.

\subsubsection{Costs}
\label{sec:costs}

When solving an OPF with dispatchable loads in \matpower, the costs of dispatchable loads is added to that of the generator production costs.
Since the objective function of the maximum loadability limit is just $\alpha$ (see \eqref{eq:maxll-obj}), we are not interested in keeping track of the cost of dispatchable loads and generators.
These costs are therefore set to zero.

The cost for $\alpha$ is defined in the callback function introduced in Section \ref{sec:defin-vari-alpha}.
It is set to -1 since the objective in \eqref{eq:maxll-obj} is to maximize $\alpha$, whereas \matpower performs a minimisation.

\subsubsection{Functions}
\label{sec:functions}

The functions in Table \ref{tab:functions} are used to set up and solving the maximum loadability limit problem with \matpower.

\begin{table}[!h]
  \centering
  \begin{tabular}{lp{8cm}}
  \toprule
  Name  & Description \\
  \midrule
  \codemat{maxloadlim} & Main function for solving the maximum loadability limit problem.\\
  \codemat{prepare\_maxll} & Prepares the \matpower case before the AC OPF is run, see Section \ref{sec:adjust-matp-case}.\\
  \codemat{userfcn\_direction\_mll\_formulation} & Callback function to define new variables, constraints and costs, see Section \ref{sec:defin-vari-alpha}. \\
  \codemat{postproc\_maxloadlim} & Post-processing function that transforms the dispatchable loads back to normal loads.\\
  \bottomrule
  \end{tabular}
  \caption{Functions used for the maximum loadability limit problem}
  \label{tab:functions}
\end{table}


\end{document}
